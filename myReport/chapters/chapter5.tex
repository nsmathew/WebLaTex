\chapter{Conclusions}

With the advancement of technology and the widespread adoption of social media, the anticipated response times for businesses to address complaints have significantly reduced. Furthermore, this evolution has introduced numerous online avenues through which customers can seek assistance or voice their grievances. With the substantial visibility offered by these platforms, organisations are exposed to the effects of negative online word of mouth. Consequently, the implementation of tools to automatically detect complaints has many benefits for organisations.\\

Building upon earlier research in this field, this study has led to a few key conclusions. First and foremost, it's apparent that BERTweet outperforms the other assessed transformer models in identifying complaints posted on Twitter. Notably, BERTweet shows comparatively good performance even when dealing with limited finetuning data. However, caution is advised when generalizing this finding to other platforms like Facebook, given the constraints of the pre-training data (limited to Twitter data). Additionally, a significant takeaway is that organizations constrained by memory and computational resources can effectively opt for smaller models like DistilBERT without significantly compromising predictive performance.\\

In conclusion, there exist several promising paths for further research within this domain. Analyzing the performance of similar models on data from other social media platforms like Facebook could yield deeper insights into the models' capabilities on linguistically diverse datasets. Another avenue involves evaluating the multi-modal approach with the BERTweet model and whether the inclusion of additional linguistic cues could enhance its performance. Lastly, considering the ongoing advancements in generative models, exploring prompting with solutions that use state-of-the-art models like GPT 3.5 / 4 could have great value.
