\chapter{Introduction}

\section{Background}
In the act of complaining, dissatisfaction or annoyance is expressed by a person or entity in response to a previous or ongoing event that has negatively impacted them \cite{olshtain_speechact_1987}. It provides an avenue to direct dissatisfaction to the appropriate organisation or individual with the hope of rectification or redressal. The event or action could be concerning a product or service procured by the concerned person or entity. The need to recognise, acknowledge and act on complaints is of significant importance to businesses and organisations to retain their customers while maintaining their reputations.

\begin{table}[ht]
    \centering
    \begin{tabularx}{\textwidth}{|X|}
        \hline
        \rowcolor[gray]{0.7}
        \textbf{Example complaints from Twitter}                                                                                                                       \\
        \hline
        hi please i cant find a driver for video card ( nvidia geforce 8500 gt ) for mac please send me a link when i can download a driver   \\
        \hline
        i wonder how many valid <user> warranties were denied today due to made-up , illegitimate policy . schools aren't commercial property \\
        \hline
        what is your policy on false advertising regarding sale items ? i was refused a sale in westfield due to a company error on pricing   \\
        \hline
        thanks to <user> ' s incompetence i now can't work till october 4th , when the ati card arrives .                                    \\
        \hline
    \end{tabularx}
    \caption{Sample complaints extracted from Twitter, exhibiting diverse degrees of complaint expression and severity. These complaints are sourced from data that has undergone the preprocessing steps outlined in Chapter 3.}
    \label{tab: ex_complaints}
\end{table}



Until the advent of online platforms and specifically social media, the impact of negative word-of-mouth was confined to a relatively limited audience. However, since then complaints posted online have the potential to rapidly go viral, reaching millions of individuals and significantly damaging a company's brand reputation and goodwill in a short period \cite{tripp_when_2011}. Customers are able to express their complaints directly, conveniently, and with enhanced effectiveness to organisations through multiple social media channels and platforms \cite{balaji_customer_2015}.
\newline \newline
In addition to the timely addressing of customer complaints, automated detection of complaints in natural language has several other purposes. Linguists could gain a more detailed understanding of the context, intent, and various types of complaints on a larger scale while psychologists could utilise this information to identify the underlying human traits that drive the behaviour and expression of complaints. Developing downstream natural language processing (NLP) applications, such as dialogue systems is another use case of this task \cite{preotiuc-pietro_automatically_2019}.
\newline \newline
Attempting to identify complaints manually through the multitude of posts and streams coming through the various social media channels is neither practical nor scalable. Various approaches to automate this task have been explored. The traditional vector-space method utilizing dictionaries has been applied in other text classification tasks \cite{liang_dictionary-based_2006}. Latent Semantic Indexing based on Singular Value Decomposition along with linguistic style features has been utilised to classify emails as complaints or not \cite{coussement_improving_2008}. In recent years, we have seen the use of various Machine learning and Natural Language Processing (NLP) based approaches for similar classification problems. The performance of logistic regression over various types of feature spaces against neural-network based models like Multi-layer Perceptron (MLP) and Long Short Term Memory (LSTM) has been analysed by \cite{preotiuc-pietro_automatically_2019} on Twitter feeds. The use of more advanced approaches using transformer networks has shown to have better results as explored by \cite{jin_complaint_2020}. As part of this paper, the use of the BERT and its many variants, including that of lightweight versions that have been created in the recent past will be assessed further on a publicly available Twitter dataset.

\section{Aims and Objectives}

**TO UPDATE**

