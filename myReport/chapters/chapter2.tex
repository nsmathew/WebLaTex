\chapter{Literature Survey}

\section{The act of complaining}

As per \cite{olshtain_speechact_1987}, the speech act of complaining in the traditional sense can be understood from the perspective of the speaker stating their displeasure or dissatisfaction to a target entity or individual. This is done as a reaction to an unfavourable event that is currently taking place or has already occurred. The authors believe a few preconditions have to be satisfied to result in a complaint being made. This includes the speaker's belief the entity or individual is responsible for the unfavourable outcome and that the speaker in question suffers from the consequences. The result is a verbally expressed complaint.
\newline \newline
This expression of complaint could be carried out in various ways. The speaker might choose to directly communicate their complaints or concerns to the individual or entity, either immediately after the incident or at a later time. Or they might voice their grievances to others through word-of-mouth or they could even opt to escalate the issue by involving a third party, such as a consumer advocacy office \cite{sparksComplainingCyberspaceMotives2010}.
\newline\newline
The authors of \cite{olshtain_speechact_1987} further delve into the intentions of the speaker in making the complaint. They argue this is carried out with either the hope of repair of the situation or as a 'Face Threatening Act' \cite{brownPolitenessUniversalsLanguage1987}, with the purpose being to damage the face of the individual or entity against whom the complaint is made. In this scenario, a face-threatening act refers to an action that challenges the reputation of the recipient by going against what the recipient desires. These acts can manifest in a verbal form including with variation in tone or inflection or using non-verbal methods.
\newline \newline
While such complaints could be considered direct complaints as per \cite{boxerSocialDistanceSpeech1993}, the authors additionally highlight the use of indirect complaining in speech. In the case of indirect complaints, the speaker does not attribute responsibility for the cause of the complaint to the individual or entity being addressed. The authors theorise, an indirect complaint is used to bring about 'solidarity' between speakers, which is contrary to the use of direct complaints. It can serve as a means to initiate conversations and establish temporary connections with others. The scope of the data for this project (described in the subsequent chapter) is primarily focused on direct complaints as they are selected based on tweets being addressed to a brand's customer service handle. However it is possible, tweets which fall into the category of indirect complaints are also included in the dataset.
\newline \newline
Analysing deeper into which types of customers complain more, \cite{sharma_complainers_2010} have looked at how personality traits like impulsivity and self-monitoring impact customer complaining behaviour. \textit{Impulsivity}, as defined by \cite{rookNormativeInfluencesImpulsive1995}, refers to a consistent inclination of customers to act spontaneously and immediately,  without much reflection or careful consideration of available options or potential consequences. This trait remains relatively stable over time for such customers. \textit{Self-monitoring} is described by  \cite{bechererSelfMonitoringModeratingVariable1978} as the propensity to adjust one's behaviour based on the actions or behaviour of others. High self-monitoring individuals are sensitive to others' expressions and behaviour, relying on social cues for their actions, while low self-monitoring individuals may be influenced by personal traits. From their experiments, \cite{sharma_complainers_2010} concluded that individuals with high impulsiveness tend to complain more than those with low impulsiveness, whereas individuals with high self-monitoring tend to complain less than those with low self-monitoring. These effects tend to be more pronounced in situations where the level of dissatisfaction is high. The level of \textit{involvement} a customer has in product or service also matters since the more deeply a customer engages with a consumption scenario, their inclination to invest resources like time, effort, and money into addressing or complaining about an unsatisfactory encounter increases \cite{lauIndividualSituationalFactors2001}. The results from \cite{sharma_complainers_2010} also validated the positive influence a consumer's degree of involvement has on their likelihood of engaging in complaining behaviour.


\section{Complaining online}
The act of complaining exists online in various forms and with varying degrees of intensity and this prevalence lead to the emergence of third-party organisations that provide online channels for customers' ease and convenience \cite{tripp_when_2011}. Notably, there are complaint websites like complaintsboard.com, review websites like trustpilot.com as well as consumer organisations' sites such as consumeraffairs.com, where customers can share their negative experiences and exchange information with others. The impact of negative word of mouth is quite high due to the ease with which negative reports can rapidly reach millions of people, potentially causing significant harm to a company's brand. Various user-generated content platforms such as YouTube, Twitter, and Facebook serve as spaces for expressing complaints. Brands use these platforms for user engagement and this provides the users with the required visibility to potentially raise or escalate an issue. With numerous such options available online, companies can experience significant repercussions arising from actions taken by dissatisfied customers \cite{tripp_when_2011}.
\newline \newline
Of the 431 online complaints assessed by \cite{tripp_when_2011}, 96\% followed what they call a double deviation. This occurs when customers experience both a product or service failure followed by multiple unsuccessful attempts to resolve the issue, resulting in them feeling they have been violated twice. Such customers then resort to online complaining. Their urge to complain online is driven by how they felt betrayed rather than simply being dissatisfied or with any form of malicious intentions to hinder business operations.
\newline\newline
Complaining online is also associated with electronic word-of-mouth or EWOM, which involves sharing information online with a wider group, and it remains accessible over an extended period while often being anonymous \cite{hennig-thurauElectronicWordofmouthConsumeropinion2004}. This type of communication can take place on various platforms, ranging from official company-sponsored sites to unaffiliated blogs. The Internet offers consumers a convenient and anonymous platform to express negative word-of-mouth by sharing their viewpoints and complaints with others. Among the different forms of EWOM, consumer reviews are particularly noteworthy, as they provide valuable insights about products, whether positive or negative \cite{sparksComplainingCyberspaceMotives2010}. Such Negative electronic WOM (EWOM), can significantly damage a brand's reputation and influence potential customers to seek alternative products or services.
\newline \newline
Technology provides an accessible channel that allows consumers to complain with significant ease, making it available to anyone with internet access, even those who may be hesitant to complain directly to the company \cite{sparksComplainingCyberspaceMotives2010}. The reviews and comments posted by consumers online can hold considerable influence over decisions made by other fellow consumers. From an organisation's perspective, the use of online complaining by consumers has some indirect negative consequences as well. The potential experience and knowledge frontline personnel could gain from addressing the complaints directly are lost and this has long-term implications for the organisation. The study by \cite{sparksComplainingCyberspaceMotives2010} on online complaining in the hospitality industry, corroborates the double deviation theory touched upon earlier. Most complaints reviewed involved the individual complaining online after having failed to receive a satisfactory resolution from the hotel staff. Another key finding was the altruistic nature of the complaints, intending to warn other potential customers of the problems. The nature of complaints points to a sense of unfairness being experienced by the guests due to their initial complaints being inadequately addressed and in some cases, combined with a lack of empathy from the staff.

\section{Complaining in social media}
With the increased penetration of social media in the lives of consumers, they now have the ability to express their grievances directly and effectively to service providers using multiple social media platforms \cite{balaji_customer_2015}. Prior to this, a significant portion of dissatisfied customers refrained from lodging complaints due to the perception that the costs associated with complaining far outweighed the benefits linked to resolution \cite{sharma_complainers_2010}. This has had a major impact at the fundamental level on how customers complain across industries. Aside from lodging a complaint, consumers also use the social media channels to publicly vent their anger, a behaviour they formerly exhibited by privately expressing dissatisfaction to friends and family.\\

The study carried out by \cite{balaji_customer_2015} investigated the elements that incite customers to participate in complaining activities via social media. They asserted that the act of complaining through social media was shaped by multiple factors. These include, perceptions of not being treated fairly and wanting retribution, attributions of causality and the personal identity and traits of the individual. Particularly, they found that distinct factors impact both private and public complaining communicated through social media. They found the level of perceived unfairness in the situation had more impact on customers choosing to publicly complain through social media. Additionally, a positive view of the firm's compensation terms fostered the expectation that complaining increased the chance of resolution. Consequently, dissatisfied customers were more inclined to express their grievances privately to the service provider instead of publicly criticising them on social media.\\

Organisations have also been promoting the use of social media and digital channels as a medium for providing customer support while moving away from conventional contact points like call centres \cite{sunDoesActiveService2021}. The cost of handling per customer on Twitter, for example, is significantly lower when compared with a call centre. It makes the situation convenient for the customer as well since they can interact with a brand via the social media handles at their preferred time rather than having to wait for a prolonged time over a phone call. Effectively managing a customer's complaint also has the potential to transform a service failure into a favourable brand encounter for the individual customer as well as showcase the brand in a positive light among other social media users.\\

Utilising data from Twitter with the brand accounts of international airlines as the basis, the study \cite{sunDoesActiveService2021} set out to determine if active service intervention of complaints online was driving further complaining on social media. They found, while  increased service activity did lead to a rise in customer complaints on social media, a improved customer service quality actually reduced future complaints. They feel if a substantial number of discontented customers, who might not otherwise use conventional contact methods, opt for social media support, it is likely to lead to reduced customer attrition for the companies over time. However, they suggest caution, since this multiplier effect can be harmful if the failure of services gains traction over the actual success cases.

\section{Self-expression on Twitter}
**TO UPDATE**

\section{Transformers}
**TO UPDATE**

\section{Ongoing research}
**TO UPDATE**