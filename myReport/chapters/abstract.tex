\chapter*{\Large \center Abstract}

A complaint is a statement made by a person or an entity with the intent to indicate something is unacceptable or unsatisfactory. This is commonly used in various aspects of day-to-day life including when conducting business operations. With the proliferation of social media across our lives and the active enablement of such platforms by organisations for user engagement, it has become a common medium for users to raise complaints. With such complaints being publicly visible, it is imperative for organisations to identify, prioritise and respond to these complaints swiftly. Automatically identifying complaints in social media is an active area of research. In the past few years, the focus has been on using NLP approaches driven by developments in transfer learning and transformer-based models.
\newline \newline
In this paper, the use of these approaches is extended by assessing variations of the BERT model, including BERTweet, which is pre-trained on Tweets and 'lightweight' models such as DistillBERT, MobileBERT and BERT tiny which are meant to reduce the time required for fine-tuning as well as inference. BERTweet performs the best with an F1 of The dataset used consists of anonymised and annotated (complaint or not) Twitter (rebranded as X) data utilized in previous research and currently available in the public domain. In addition, the act of complaining and its nature when used online and in social media are analysed from a linguistic perspective along with discussions on state-of-the-art approaches for such NLP tasks.
\newline \newline
