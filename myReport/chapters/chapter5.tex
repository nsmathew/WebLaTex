\chapter{Conclusions}

With the advancement of technology and the widespread adoption of social media, the anticipated response times for businesses to address complaints have significantly reduced. Furthermore, this evolution has introduced numerous online avenues through which customers can seek assistance or voice their grievances. These platforms grant substantial visibility, thereby exposing organizations to the effects of negative online word of mouth. Consequently, the implementation of tools to automatically detect complaints has many benefits for organisations.\\

This study has progressed from previous research on this topic and it has allowed for several conclusions to be made. One is that BERTweet holds an advantage over the other transformer models tested for identifying complaints in Twitter feeds. However, caution is advised over extending this conclusion for other feeds such as that from Facebook due to the nature of pre-training data used (restricted to Twitter data). Even in situations with limited data for finetuning BERTweet provides comparatively good performance. The other important conclusion is how organisations with memory and compute constraints, can use smaller models like DistilBERT without a major hit to the predictive performance. 

There are also several avenues for future research on this topic. Behaviour of similar models on other social media data such as Facebook could provide more insight on if models can perform competitively on more linguistically diverse datasets. Another potential are of interest could be on using the multi-modal approach with BERTweet to understand if an injection of additional linguistic cues could help the model perform better. Finally with the generative models being an active area of research the use of prompting with the state-of-the-art GPT could also be explored as a potential solution.
