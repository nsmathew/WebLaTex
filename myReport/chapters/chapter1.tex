\chapter{Introduction}

\section{Background}

In the act of complaining, dissatisfaction or annoyance is expressed by a person or entity in response to a previous or ongoing event that has negatively impacted them \cite{olshtain_speechact_1987}. It provides an avenue to direct dissatisfaction to the appropriate organisation or individual with the hope of rectification or redressal. The event or action could be concerning a product or service procured by the concerned person or entity. For an organisation, the need to recognise, acknowledge and act on complaints is of significant importance to businesses and organisations to retain their customers while maintaining their reputations.
\newline \newline
Until the advent of online platforms and specifically social media, the impact of negative word-of-mouth was confined to a relatively limited audience. However, since then complaints posted online now have the potential to rapidly go viral, reaching millions of individuals and significantly damaging a company's brand reputation and goodwill in a short period \cite{tripp_when_2011}. Customers are able to express their complaints directly, conveniently, and with enhanced effectiveness to organisations through multiple social media channels and platforms \cite{balaji_customer_2015}.
\newline \newline
In addition to the timely addressing of customer complaints, automated detection of complaints in natural language has a number of other purposes. Linguists could gain a more detailed understanding of the context, intent, and various types of complaints on a larger scale while psychologists could utilise this information to identify the underlying human traits that drive complaint behavior and expression. Developing downstream natural language processing (NLP) applications, such as dialogue systems is another use case \cite{preotiuc-pietro_automatically_2019}.



\section{Aims and Objectives}

Lorem ipsum dolor sit amet, consectetuer adipiscing elit. Aenean c

\section{Overview of the Report}

Lorem ipsum dolor sit amet, 
