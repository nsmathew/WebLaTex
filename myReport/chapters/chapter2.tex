\chapter{Literature Survey}

\section{The act of complaining}

As per \cite{olshtain_speechact_1987}, the speech act of complaining in the traditional sense can be understood from the perspective of the speaker stating their displeasure or dissatisfaction to a target entity or individual. This is done as a reaction to an unfavourable event that is currently taking place or has already occurred. The authors believe a few preconditions have to be satisfied to result in a complaint being made. This includes the speaker's belief the entity or individual is responsible for the unfavourable outcome and that the speaker in question suffers from the consequences. The result is a verbally expressed complaint. 
The authors further delve into the intentions of the speaker in making the complaint. They argue this is carried out with either the hope of repair of the situation or as a 'Face Threatening Act' with the purpose being to damage the face of the individual or entity against whom the complaint is made.
\newline \newline
While such complaints could be considered direct complaints as per \cite{boxerSocialDistanceSpeech1993}, they additionally highlight the use of indirect complaining in speech. In the case of indirect complaints, the speaker does not attribute responsibility for the cause of the complaint to the individual or entity being addressed. The authors theorise, an indirect complaint is used to bring about 'solidarity' between speakers, which is contrary to the use of direct complaints. It can serve as a means to initiate conversations and establish temporary connections with others. The scope of the data for this project (described in the subsequent chapter) is primarily focused on direct complaints as they are selected based on tweets being addressed to a brand's customer service handle. But it is possible there are tweets which fall into the category of indirect complaints.
\newline \newline
Analysing deeper into which types of customers complain more, \cite{sharma_complainers_2010} have looked at how personality traits like impulsivity and self-monitoring impact customer complaining behaviour. \textit{Impulsivity}, as defined by \cite{rookNormativeInfluencesImpulsive1995}, refers to a consistent inclination of customers to act spontaneously and immediately,  without much reflection or careful consideration of available options or potential consequences. This trait remains relatively stable over time for such customers. \cite{bechererSelfMonitoringModeratingVariable1978} defines \textit{self-monitoring} as the propensity to adjust one's behaviour based on the actions or behaviour of others. High self-monitoring individuals are sensitive to others' expressions and behaviour, relying on social cues for their actions, while low self-monitoring individuals may be influenced by personal traits. 
From their experiments, \cite{sharma_complainers_2010} concluded that individuals with high impulsiveness tend to complain more than those with low impulsiveness, whereas individuals with high self-monitoring tend to complain less than those with low self-monitoring. However, these effects are more pronounced in situations where the level of dissatisfaction is high. 


\section{Complaining online}
The act of complaining exists online in various forms and with varying degrees of intensity and this prevalence lead to the emergence of third-party organisations that provide online channels for customers' ease and convenience \cite{tripp_when_2011}. Notably, there are complaint websites like complaintsboard.com, review websites like trustpilot.com as well as consumer organisations' sites such as consumeraffairs.com, where customers can share their negative experiences and exchange information with others. The impact of negative word of mouth is quite high due to the ease with which negative reports can rapidly reach millions of people, potentially causing significant harm to a company's brand. Various user-generated content platforms such as YouTube, Twitter, and Facebook serve as spaces for expressing complaints. Brands use these platforms for user engagement and this provides the users with the required visibility to potentially raise or escalate an issue. With numerous such options available online, companies can experience significant repercussions arising from actions taken by dissatisfied customers \cite{tripp_when_2011}.
\newline \newline
Of the 431 online complaints assessed by \cite{tripp_when_2011}, 96\% followed what they call a double deviation. This occurs when customers experience both a product or service failure followed by multiple unsuccessful attempts to resolve the issue, resulting in them feeling they have been violated twice. Such customers then resort to online complaining. Their urge to complain online is driven by how they felt betrayed rather than simply being dissatisfied or with any form of malicious intentions to hinder business operations.

\section{Complaining in social media}
**TO UPDATE**

\section{Self-expression on Twitter}
**TO UPDATE**

\section{Transformers}
**TO UPDATE**

\section{Ongoing research}
**TO UPDATE**