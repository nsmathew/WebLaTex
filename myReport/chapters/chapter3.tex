\chapter{Methodology}

\section{Task}
For a short text segment, $T = \{t_1, t_2, ..., t_n\}$ where $t_i$ is defined as a token, classify if the sequence of tokens $T$ is a complaint or not.

\section{Data and pre-processing}
The data used for the experiments is from Twitter. Twitter provides a good representation of social media text due to the direct connection consumers have with organisations and brands as well as the ability to express oneself \cite{preotiuc-pietro_automatically_2019}. **Add content on why Twitter**
\newline \newline
The data set created by \cite{preotiuc-pietro_automatically_2019} and further used by \cite{jin_complaint_2020} is utilised for this project. The original process for collection and annotation employed by them is breifly described below. The particular version \footnote{The data can be found here - https://archive.org/details/complaint\_severity\_data} used for the experiments is the one enhanced by \cite{jinModelingSeverityComplaints2021} with the addition of labels for the severity of complaints. These additional labels are not used for the experiments in this project.
\subsection{Domains and organisations}
A cross-industry representative collection of 93 customer service handles of organisations on Twitter were identified manually. These handles were then categorised into 9 domains based on their industry type. Since an organisation could have business activities across domains, the assigned domain was based on the products or services receiving the most number of complaints. All the domains used in the experiments are listed in Table \ref{tab: domains}. 
\begin{table}[ht]
    \centering
    \begin{tabularx}{\textwidth}{|X|c|c|c|}
    \hline
    \rowcolor[gray]{0.7}
    \textbf{Domains} & \textbf{Complaints} & \textbf{Non-Complaints} & \textbf{Total Tweets}\\
    \hline
    Food \& Beverage & 95 (73\%) & 35 (27\%) & 130 (7\%)\\
    \rowcolor[gray]{0.9}
    Apparel & 141 (55\%) & 117 (45\%) & 258 (13\%)\\
    Retail & 124 (62\%)& 75 (38\%) & 199 (10\%)\\
    \rowcolor[gray]{0.9}
    Cars & 67 (73\%)& 25 (27\%) & 92 (4\%)\\
    Services  & 207 (61\%)& 130 (39\%) & 337 (17\%)\\
    \rowcolor[gray]{0.9}
    Software \& Online Services & 189 (65\%)& 103 (35\%) & 292 (15\%)\\
    Transport & 139 (56\%)& 109 (44\%) & 248 (12\%)\\
    \rowcolor[gray]{0.9}
    Electronics & 174 (61\%)& 112 (39\%) & 286 (15\%)\\
    Other & 96 (79\%)& 33 (21\%) & 129 (7\%)\\
    \hline
    \rowcolor[gray]{0.9}
    \textbf{Total} & 1232 (63\%)& 739 (37\%) & 1971\\
    \hline
    \end{tabularx}
    \caption{The nine domains and the distribution of tweets that are complaints and those that are not. The percentages indicate how the splits are distributed.}    
    \label{tab: domains}
\end{table}  

\subsection{Collection}
The data was extracted from Twitter via the Twitter API \footnote{https://developer.twitter.com/en}. The most recent 3,200 tweets at the time of the collection exercise were extracted and the original tweets to which the customer service handles responded were identified. Then, random sampling equally for each handle, 1,971 tweets were identified where there was a response from the support's handle. To ensure a more balanced and diverse dataset, 1,478 randomly sampled tweets were added to the dataset. 739 tweets were replies to other handles (outside the 93 identified) and the remaining 739 tweets were not addressed to any Twitter handle. Table \ref{tab: tweet_counts} shows the breakdown of the total population of the tweets dataset. Tweets were filtered for English using langid.py \cite{luiLangidPyOfftheshelf2012}. Retweets were excluded and all usernames and URLs were anonymised and replaced with placeholder tokens.
\begin{table}[ht]
    \centering
    \begin{tabularx}{\textwidth}{|X|c|c|c|}
    \hline
    \rowcolor[gray]{0.7}
    \textbf{Colllection Criteria} & \textbf{Complaints} & \textbf{Non-Complaints} & \textbf{Total Tweets} \\
    \hline
    Addressed to and replied by the identified 93 customer service handles & 1239 ( 63\%) & 739 (37\%) & 1971 (58\%) \\
    \hline
    Addressed to other customer service handles & 0 & 739 (100\%)& 739 (21\%) \\
    \hline
    Not addressed to any Twitter handle & 0 & 739 (100\%)& 739 (21\%) \\
    \hline    
    \rowcolor[gray]{0.9}
    Total & 1232 (36\%) & 2217 (64\%) & 3449 \\
    \hline
    \end{tabularx}
    \caption{Selection of tweets based on random sampling and where they have received replies when addressed to the 93 customer service handles combined with random sampled tweets that are addressed to other handles and tweets that are not addressed to any handle.}    
    \label{tab: tweet_counts}
\end{table}  

\subsection{Annotation}
The classification of the 1,971 tweets as complaints or not was carried out using a binary annotation task (complaint or not). Since tweets are concise and typically express a single idea, an entire tweet was classified as a complaint if it contained at least one speech act of complaining. To guide the annotation process, a complaint definition from \cite{olshtain_speechact_1987}, stating that a complaint portrays a situation that contradicts the writer's positive expectation was used. Two of the authors with extensive annotation experience in linguistics independently labelled the 1,971 tweets. They had substantial agreement \cite{artsteinInterCoderAgreementComputational2008} with Cohen's Kappa of $\kappa$ = 0.731. In the end, 1,232 tweets (63\%) and 739 tweets (37\%) were identified as complaints and non-complaints. Table \ref{tab: domains} gives the breakdown of the complaint and non-complaint tweets for each domain.




\section{Models and Libraries}

Lorem ipsum dolor sit amet, consectetuer adipiscing elit. Aenean commodo ligula eget dolor. Aenean massa. Cum sociis natoque penatibus et magnis dis parturient montes, nascetur ridiculus mus. Donec quam felis, ultricies nec, pellentesque eu, pretium quis, sem. Nulla consequat massa quis enim. Donec pede justo, fringilla vel, aliquet nec, vulputate eget, arcu. In enim justo, rhoncus ut, imperdiet a, venenatis vitae, justo. Nullam dictum felis eu pede mollis pretium. Integer tincidunt. Cras dapibus. Vivamus elementum semper nisi. Aenean vulputate eleifend tellus. Aenean leo ligula, porttitor eu, consequat vitae, eleifend ac, enim. Aliquam lorem ante, dapibus in, viverra quis, feugiat a, tellus. Phasellus viverra nulla ut metus varius laoreet. Quisque rutrum. Aenean imperdiet. Etiam ultricies nisi vel augue. Curabitur ullamcorper ultricies nisi. Nam eget dui. 

\section{Nested cross validation}
** To UPDATE **

\section{Domain splits with nested cross validation}
** To UPDATE **

\section{Ethical, Professional and Legal Issues}

Lorem ipsum dolor sit amet, consectetuer adipiscing elit. Aenean commodo ligula eget dolor. Aenean massa. Cum sociis natoque penatibus et magnis dis parturient montes, nascetur ridiculus mus. Donec quam felis, ultricies nec, pellentesque eu, pretium quis, sem. Nulla consequat massa quis enim. Donec pede justo, fringilla vel, aliquet nec, vulputate eget, arcu. In enim justo, rhoncus ut, imperdiet a, venenatis vitae, justo. Nullam dictum felis eu pede mollis pretium. Integer tincidunt. Cras dapibus. Vivamus elementum semper nisi. Aenean vulputate eleifend tellus. Aenean leo ligula, porttitor eu, consequat vitae, eleifend ac, enim. Aliquam lorem ante, dapibus in, viverra quis, feugiat a, tellus. Phasellus viverra nulla ut metus varius laoreet. Quisque rutrum. Aenean imperdiet. Etiam ultricies nisi vel augue. Curabitur ullamcorper ultricies nisi. Nam eget dui. Etiam rhoncus. Maecenas tempus, tellus eget condimentum rhoncus, sem quam semper libero, sit amet adipiscing sem neque sed ipsum. Nam quam nunc, blandit vel, luctus pulvinar, hendrerit id, lorem. Maecenas nec odio et ante tincidunt tempus. Donec vitae sapien ut libero venenatis faucibus. Nullam quis ante. Etiam sit amet orci eget eros faucibus tincidunt. Duis leo. Sed fringilla mauris sit amet nibh. Donec sodales sagittis magna. Sed consequat, leo eget bibendum sodales, augue velit cursus nunc.
