\chapter{Literature Survey}

\section{The act of complaining}

In a traditional sense, as per \cite{olshtain_speechact_1987}, the speech act of complaining can be understood from the perspective of the speaker stating their displeasure or dissatisfaction to a target entity or individual. This is done as a reaction to an unfavourable event that is currently taking place or has already occurred. To result in a complaint being made, the authors believe a few preconditions have to be satisfied. This includes the speaker's belief the entity or individual is responsible for the unfavourable outcome and the speaker suffering from the consequences. The result is a verbally expressed complaint. 
The authors further delve into the intentions of the speaker in making the complaint. They argue this is carried out with either the hope of repair of the situation or as a 'Face Threatening Act' with the purpose being to damage the face of the individual or entity against whom the complaint is made. 
\newline \newline
Analysing further into which types of customers complain more, \cite{sharma_complainers_2010} have looked at how personality traits like impulsivity and self-monitoring impact customer complaining behaviour. Impulsivity as defined by \cite{rookNormativeInfluencesImpulsive1995}, refers to a consistent inclination of customers to act spontaneously and immediately,  without much reflection or careful consideration of available options or potential consequences. This trait remains relatively stable over time for such customers. \cite{bechererSelfMonitoringModeratingVariable1978} defines self-monitoring as the inclination to adjust one's behaviour based on the actions or behaviour of others. High self-monitoring individuals are sensitive to others' expressions and behaviour, relying on social cues for their actions, while low self-monitoring individuals may be influenced by personal traits. 
From their experiments, \cite{sharma_complainers_2010} concluded that individuals with high impulsiveness tend to complain more than those with low impulsiveness, whereas individuals with high self-monitoring tend to complain less than those with low self-monitoring. However, these effects are more pronounced in situations where the level of dissatisfaction is high.


\section{Complaining online}

The act of complaining exists online in various forms and with varying degrees of intensity and this widespread leading to the emergence of third-party organisations that provide online channels for customers' ease and convenience \cite{tripp_when_2011}. Notably, there are complaint websites like complaintsboard.com, review websites like trustpilot.com as well as consumer organisations' sites such as consumeraffairs.com, where customers can share their negative experiences and exchange information with others. The impact of negative word of mouth is quite high due to the ease with which negative reports can rapidly reach millions of people, potentially causing significant harm to a company's brand. Various user-generated content platforms such as YouTube, Twitter, and Facebook serve as spaces for expressing complaints. With numerous options available online, companies can experience significant repercussions arising from actions taken by dissatisfied customers \cite{tripp_when_2011}.
\newline \newline
In the research in \cite{tripp_when_2011}, of the 431 online complaints assessed, 96\% followed what they call a double deviation. This occurs when customers experience both a product or service failure followed by multiple unsuccessful attempts to resolve the issue, resulting in them feeling they have been violated twice. The urge for such customers to resort to online complaining is driven by how they felt betrayed rather than simply being dissatisfied or with malicious intentions to hinder business operations.




