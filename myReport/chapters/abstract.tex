\chapter*{\Large \center Abstract}

A complaint is a statement made by a person or an entity with the intent to indicate something is unacceptable or unsatisfactory. This is commonly used in various aspects of day-to-day life including for conducting business operations. With the proliferation of social media across our lives it has become a common medium for users to raise complaints. With such complaints being publicly visible, it is imperative for organisations to identify and respond to these complaints swiftly. 
Automatically identifying complaints in social media is an active area of research and in the past few years, the focus has been on using NLP approaches driven by recent developments in transformer-based models. \newline\newline
In this paper, this approach is extended by exploring lightweight transformer based models which are meant to reduce the time required for fine-tuning as well as inference.  The performance of these 'lightweight' models is compared with the traditional transformer models for this particular task. The data used will be anonymised Twitter data that has been used in previous research and is available in the public domain. In addition, the nature of complaints will be analysed from a linguistic perspective and the current state-of-the-art approaches for such NLP tasks will be discussed.