\chapter{Conclusions}

With the advancement of technology and the widespread adoption of social media, the anticipated response times for businesses to address complaints have significantly reduced. Furthermore, this evolution has introduced numerous online avenues through which customers can seek assistance or voice their grievances. These platforms grant substantial visibility, thereby exposing organizations to the effects of negative online word of mouth. Consequently, the implementation of tools to automatically detect complaints has many benefits for organisations.\\

This study has progressed from previous research on this topic and it has allowed for several conclusions to be made. One is that BERTweet holds an advantage over the other transformer models tested for identifying complaints in Twitter feeds. However, caution is advised over extending this conclusion for other feeds such as that from Facebook due to the nature of pre-training data used (restricted to Twitter data). Even in situations with limited data for finetuning BERTweet provides comparatively good performance. The other important conclusion is how organisations with memory and compute constraints, can use smaller models like DistilBERT without a major hit to the predictive performance. 

Building upon earlier research in this field, this study has led to a few key conclusions. First and foremost, it's apparent that BERTweet outperforms the other assessed transformer models in identifying complaints within Twitter streams. Nevertheless, it's advisable to exercise caution when generalizing this finding to other platforms like Facebook, given the constraints of the pre-training data (limited to Twitter data). Notably, BERTweet shows comparatively good performance even when dealing with limited finetuning data. Additionally, a significant takeaway is that organizations constrained by memory and computational resources can effectively opt for smaller models like DistilBERT without significantly compromising predictive performance.

In conclusion, there exist several promising paths for further exploration within this research domain. Analyzing the performance of similar models on alternative social media platforms like Facebook could yield deeper insights into the models' capabilities on linguistically diverse datasets. Another avenue of potential interest involves evaluating the application of the multi-modal approach with the BERTweet model, examining whether the inclusion of additional linguistic cues enhances its performance. Lastly, considering the ongoing advancements in generative models, exploring prompting with solutions that use state-of-the-art models like GPT 3.5 and GPT 4 as potential classification systems is 
